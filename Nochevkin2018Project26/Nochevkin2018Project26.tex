\documentclass[12pt, a4paper]{article}

%% Language and font encodings
\usepackage[T2A]{fontenc}
\usepackage[utf8]{inputenc}
\usepackage[russian, english]{babel}

%% Sets page size and margins
\usepackage[a4paper,top=3cm,bottom=2cm,left=3cm,right=3cm,marginparwidth=1.75cm]{geometry}

%% Useful packages
\usepackage{amsmath}
\DeclareMathOperator*{\argmax}{argmax}
\DeclareMathOperator*{\argmin}{argmin}
\usepackage{graphicx}
%\usepackage[colorinlistoftodos]{todonotes}
\usepackage[colorlinks=true, allcolors=blue]{hyperref}

\begin{document}
\selectlanguage{russian}
\title{\textbf{Определение местоположения по сигналам акселерометра}}
\author{\bf \em Zaynulina E. T., Kiseleva E. A., Protasov V. P., Fateev D. A.,\\
 \bf \em Bozhedomov N., Tolkanev A. A., Nochevkin V., Ryabov A. }
\date{}
\maketitle


\begin{abstract}
	Данная статья посвящена использованию методов машинного обучения в задаче определения
местоположения,траектории движения и создания на их основе карты по результатам работы устройств, не основанных на gps приборах. Задача является актуальной и имеет различные практические применения, в частности детектирование деятельности человека в отсутвии сигнала gps. Поставленная задача решается по сигналам датчика телефона – акселерометра,гироскопа,магнетометра и ... ,встроенных в мобильное устройство движущегося человека. Для этих целей решаются проблемы негативного влияния шумового загрязнения(не уверен в формулировке),коррекции ошибок на основе новых данных и ... .

{\bf Новизна:} задача исследования ставится в терминах projection to the latent space.

{\bf Ключевые слова:}  обработка сигналов; сенсоры; акселерометр; анализ данных; машинное обучение; фильтр Кальмана,магнетометр.
\end{abstract}



\setcounter{secnumdepth}{0}
\section{Введение}



\begin{thebibliography}{4}
\end{thebibliography}

\end{document}
