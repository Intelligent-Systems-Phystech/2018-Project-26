\documentclass[12pt,twoside]{article}
\usepackage{jmlda}
%\NOREVIEWERNOTES
\title
    [Определение местоположения по сигналам акселерометра] % Краткое название; не нужно, если полное название влезает в~колонтитул
    {Определение местоположения по сигналам акселерометра}
\author
    [8 студентов МФТИ] % список авторов для колонтитула; не нужен, если основной список влезает в колонтитул
    {Божедомов~Н.\,И., 
Зайнулина~Э.\,Т.,
Киселёва~Е.\,А.,
Ночевкин~В.\,В.,
Протасов~В.\,П.,
Рябов~А.,
Толканев~А.\,А.,
Фатеев~Д.\,А.\newline} % основной список авторов, выводимый в оглавление
\thanks
    {Задачу поставили:  Гарцеев~И.,\,Стрижов~В.\,В.
    Консультант:  Мотренко~А.\,П.}
\email
    {nikita.bozhedomov@gmail.com,
zaynulina.et@phystech.edu,
kiseleva.ea@phystech.edu,
nochevkin@phystech.edu,
izakladno@yandex.ru,
ryabov.alexandr@phystech.edu,
artem.tolkanev@phystech.edu,
fateev.da@phystech.edu\newline}
\organization
    {МФТИ(ГУ)}
\abstract
    {В работе предложен метод восстановления траектории движения тела и определения текущего местоположения на основе данных движения тела и датчиков, таких как акселерометр, гироскоп и магнитометр, а так же начального положения тела.  Метод расширяет и обобщает предыдущие решения по определению траекторий движения тела по данным акселерометра, используя PLS regression анализ и дополнительные источники данных, полученные с датчиков, перечисленных выше.

\bigskip
\textbf{Ключевые слова}: \emph {Инерциальная навигация, акселеромет, PLS regression}.}

\begin{document}
\maketitle
%\linenumbers
\section{Введение}
Решается задача определения положения расположения тела без использования GPS. Работа актуальна, так как существуют ситуации, когда связь с внешним миром по каким-либо причинам может отсутствовать, есть необходимость определения траектории и положения тела на основе инерциальных датчиков. Так же позиционирование по GPS не всегда является точным. По похожей теме, например определение активности человека уже есть работы \b{[1]}. \\<--------------- не сделал ----------------->

\section{Описание}
В работе введены векторы состояния динамической системы исследуемого тела, один из которых получен по данным датчиков (вектор $\vec X_t$, $t - $ момент времени описания системы), а второй получен из предыдущего посредством использования фильтра Калмана (вектор $\vec Y_t = K(\vec X_t,\vec X_{t-1})$). На основе подпространства состояний системы (используются только данные акселерометра) получена кусочно-гладкая функция описывающая траекторию движения тела $\vec F(\vec Y_{accelerometer})$. Результат работы представлен оператором $L(\vec F, Y_{gyroscope, magnetometer})$ над $\vec F(\vec Y_{accelerometer})$, корректирующим траекторию движения тела, используя данные гироскопа и магнитометра. Прогноз получен методом PLS regression. \\<------------------ не сделал ------------------>



\section{Ссылки}
$[1]$ $Deep\ Learning\ for\ Sensor-based\ Activity\ Recognition:\ A\ Survey.\ $\\
https://arxiv.org/pdf/1707.03502.pdf\\
\\$[2]$ $LSTMs\ for\ Human\ Activity\ Recognition$ \\ https://github.com/guillaume-chevalier/LSTM-Human-Activity-Recognition\\
\\$[3]$ $ $SmartPDR:\ Smartphone-Based\ Pedestrian\ Dead\
Reckoning\ for\ Indoor\ Localization$\\https://ieeexplore.ieee.org/stamp/stamp.jsp?tp=&arnumber=6987239&tag=1


\section{Датасеты}
$[1]$ $A\ public\ domain\ dataset\ for\ human\ activity\ recognition\ using\ smartphones$ \\https://upcommons.upc.edu/handle/2117/20897\\
\\$[2]$ Свой датасет от квадракоптера с GPS, магнитометром, акселерометром, гироскопом, магнитометром, камерой, блекджеком \\<------------------ in process... ---------------->

\section{Чтобы было}
$[1]$ Partial least squares regression
and projection on latent structure
regression (PLS Regression)\\https://www.utdallas.edu/~herve/abdi-wireCS-PLS2010.pdf\\
\\$[2]$ Partial Least Squares (PLS) methods for neuroimaging: A tutorial and review \\https://www.utdallas.edu/~herve/abdi-kwmaPLS4NeuroImage2010.pdf\\
\\$[3]$ LSTMs for Human Activity Recognition\\https://github.com/guillaume-chevalier/LSTM-Human-Activity-Recognition\\
\section{Датчики для квадракоптера}
- акселерометр $GY-61\ XC4478\ ADXL335$\\
- $VC0706\ UART\ Camera$ с сенсором $MT9V011$ на схеме $IM130517001_VC070$\\
- барометр $BMP180$\\
- кардридер $MH-SD\ Card\ Module\ MOD-1142$ 
\end{document}
