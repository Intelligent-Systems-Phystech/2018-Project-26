%%%%%%%%%%%%%%%%%%%%%%%%%%%%%%%%%%%%%%%%%%%%%%%%%%%%%%%%%%%%%%%%%%%%%%%%%%%%%%%%
%2345678901234567890123456789012345678901234567890123456789012345678901234567890
%        1         2         3         4         5         6         7         8

\documentclass[letterpaper, 10 pt, conference]{ieeeconf}  % Comment this line out
                                                          % if you need a4paper
%\documentclass[a4paper, 10pt, conference]{ieeeconf}      % Use this line for a4
                                                          % paper

\IEEEoverridecommandlockouts                              % This command is only
                                                          % needed if you want to
                                                          % use the \thanks command
\overrideIEEEmargins
% See the \addtolength command later in the file to balance the column lengths
% on the last page of the document



% The following packages can be found on http:\\www.ctan.org
%\usepackage{graphics} % for pdf, bitmapped graphics files
%\usepackage{epsfig} % for postscript graphics files
%\usepackage{mathptmx} % assumes new font selection scheme installed
%\usepackage{times} % assumes new font selection scheme installed
%\usepackage{amsmath} % assumes amsmath package installed
%\usepackage{amssymb}  % assumes amsmath package installed

\usepackage{cmap}					% поиск в PDF
\usepackage{mathtext} 				% русские буквы в формулах
\usepackage[T2A]{fontenc}			% кодировка
\usepackage[utf8]{inputenc}			% кодировка исходного текста
\usepackage[english,russian]{babel}	% локализация и переносы
\usepackage {algorithmic}
%\usepackage []{algorithm2e}
%\usepackage {algorithmicx}
%\usepackage {program}

\title{\LARGE \bf
Определение местоположения по сигналам акселерометра
}

%\author{ \parbox{3 in}{\centering Huibert Kwakernaak*
%         \thanks{*Use the $\backslash$thanks command to put information here}\\
%         Faculty of Electrical Engineering, Mathematics and Computer Science\\
%         University of Twente\\
%         7500 AE Enschede, The Netherlands\\
%         {\tt\small h.kwakernaak@autsubmit.com}}
%         \hspace*{ 0.5 in}
%         \parbox{3 in}{ \centering Pradeep Misra**
%         \thanks{**The footnote marks may be inserted manually}\\
%        Department of Electrical Engineering \\
%         Wright State University\\
%         Dayton, OH 45435, USA\\
%         {\tt\small pmisra@cs.wright.edu}}
%}

\author{Зайнулина~Э.\,Т., Киселёва~Е.\,А.,Фатеев~Д.\,А.,
    Божедомов~Н.,
    Протасов~В.% <-this % stops a space
\thanks
    {
   Научный руководитель:  Стрижов~В.\,В.
    Консультант:  Мотренко~А.}
    }


\begin{document}



\maketitle
\thispagestyle{empty}
\pagestyle{empty}


%%%%%%%%%%%%%%%%%%%%%%%%%%%%%%%%%%%%%%%%%%%%%%%%%%%%%%%%%%%%%%%%%%%%%%%%%%%%%%%%
\begin{abstract}
Cистемы внутреннего и наружного позиционирования играют важную роль в современном мире. Задача определения места положения хорошо решена Глобальными системами позиционирования (GPS); однако, не всегда есть возможность воспользоваться ими.

В данной статье предложен метод увеличения точности отслеживания человека по сигналам акселерометра, гироскопа и магнитометра.

Чтобы увеличить точность и уменьшить время вычислений, мы используем модель PDR (pedestrian dead reckoning). Так как данные неизбежно собираются с некоторым шумом, применяем фильтр Калмана. Задача исследования ставится в терминах Projection to Latent Spaces, т.к. восстановление траектории происходит целиком, а не от точки к точке.
\end{abstract}
\begin{keywords}
PDR, PLS, Position system, User location.
\end{keywords}

%%%%%%%%%%%%%%%%%%%%%%%%%%%%%%%%%%%%%%%%%%%%%%%%%%%%%%%%%%%%%%%%%%%%%%%%%%%%%%%%
\section{ВВЕДЕНИЕ}
Определение местонахождения человека черезвычайно важная задача. Мы пользуемся системами GNNS, чтобы не потерться в городе или добраться до соседнего, обеспечить безопасность в толпе, координировать действия рабочих. Но GNNS определяет местоположение в здании неточно.

Методы основанные на сигналах WiFi, Bluetooth, UWB [7][12] требуют дополнительной информации: карты помещения или набора ключевых точек. Это не всегда удобно или даже реализуемо.

С другой стороны сейчас у каждого человека в кармане целый арсенал средств для ориентирования в пространстве. Все спрятано в наших смартфонах: акселерометр, гироскоп, магнитометр. В нашей работе мы предлагаем восстанавливать полную траекторию по сигналам этих датчиков, используя модель PDR, избавляясь от зашумленности с помощью фильтра Калмана.


\section{ПОСТАНОВКА ЗАДАЧИ}

\subsection{Модель}

$$ \mathrm{f}  : \mathrm{X}   \rightarrow \mathrm{Y}$$

$\mathrm{X} \in \mathbb{R}^{N \times T} $ - матрица признаков, составленная из векторов, сопоставленных каждому временному ряду: $\mathrm{s(l)} \in \mathbb{R} ^ {N} $

$\mathrm{Y} \in \mathbb{R}^{2 \times T}$  - траектория пешехода

$y(t)  $ - строки матрицы $\mathbb{Y}$,  временные ряды, описывающие изменения глобальных координатпешехода во времени.

\subsection{Подзадачи}

\begin{enumerate}
\item Определение класса местоположения датчика: рука, нога, сумка, тело.
\item  Предсказании траектории на основе полученного класса\[f \to f_1 ~ f_2\]
\[f_1: X \to P = \{0, 1, 2, 3\}\]
\[f_2: X, ~P \to Y\]
\end{enumerate}

\subsection{Методы}

Используем метод опорных векторов для классификации и регрессии.

Минимизируем $S(w|f, X, Y)$:

\[\min_{w, w_0}S(w, w_0) = \|w\|^2+C\sum_{i}\xi_i\]
\[subject~to~y_i(w^Tx_i+w_0)\geq 1-\xi_i\]
\[\xi_i \geq 0~\forall i\] 

$S(w, w_0)$ - штраф за суммарную ошибку.

\subsection{Оценка качества модели}

Используем критерий суммы квадратов отклонений предсказанных координат от истинных, корреляция между предсказанной и истинной траекториями пешехода.

\[w^* = \arg\min_{w}S(w|f, X, Y).\]


\section{БАЗОВАЯ МОДЕЛЬ}
\begin{itemize}

\item Чтобы устранить зашумленность данных применяем Гауссовское сглаживание для 6 каналов гиростабилизатора и 2 скоростных каналов. Полученные скорости преобразуем в вектор признаков.
\item Строим матрицу признаков X.
\item С помощью матрицы X и SVM-классификатора определяем, где находится датчик. (Рука, нога, сумка, тело).
\item Для каждого класса обучаем 2 SVMR-регрессора. Первый определяет скорости движения человека в разные промежутки времени. Но эти результаты имеют ошибку из-за неточности датчиков гироскопа, акселерометра. Решаем задачу минимизации этой ошибки.

\[\min_{\{x^1_I, x^51_I,\dots\}}V_{bias}=
\min_{\{x^1_I, x^51_I,\dots\}}\sum_{f \in F_2}\|v_C^F-v_R^f\|+
\lambda\sum_{f \in F_1}\|x^f_I\|^2,\]
\[v_C^f = R_{SW}^f\sum_{f'=1}^f R_{WI}^{f'}(a_I^{f'}+x_I^{f'}),\]
где $f$ - единица блока выборки, $F$ - блок выборки, $v_C^F$ - скорректированное значение скорости, $v_R^f$ - предсказанное значение скорости, $I$ - система координат устройства, $W$ - глобальная система координат, $S$ - IMU-стабилизированная система координат, $R_{AB}$ - матрица перехода из системы координат $B$ в систему координат $A$.

Второй предсказывает угловые скорости.

\item По полученным скоростям восстанавливаем траекторию пешехода.
\item Алгоритм:

\begin{algorithmic}[1]
\REQUIRE $X, Y_{class}, Y, X_{test}$
\STATE $initialize ~ classifier\_options$
\STATE $classifier = SVMClassifier(classifier\_options);$
\STATE $classifier.fit(X, Y_{class})$
\FOR {$cls ~ in ~ classes$:}
\STATE $initialize ~ regressor\_cls\_optons$
\STATE $regressor\_cls = SVRRegressor(regressor\_cls\_optons)$
\STATE $regressor\_cls.fit(X[X[ind] \in cls], Y)[Y[ind] \in cls])$
\ENDFOR
\STATE $Y_{test-class} = classifier.predict(X_{test})$
\FOR {$cls ~ in ~ classes$:}
\STATE $Velocity\_cls = regressor\_cls.predict(X_{test} [Velocity\_class[ind] == cls]$
\STATE $x^1_I, x^51_I,\dots = \argmin_{\{x^1_I, x^51_I,\dots\}}V_{bias}\_cls$
\STATE $Velocity\_cls = R_{SW}^f\sum_{f'=1}^f R_{WI}^{f'}(a_I^{f'}+x_I^{f'})$
\STATE $Trajectory\_cls$ recovery depending on $Velocity\_cls$
\ENDFOR
\RETURN $Full\_trajectory$
\end{algorithmic}

\end{itemize}


\section{БАЗОВЫЙ ЭСПЕРИМЕНТ}




\section{ВЫВОДЫ}







\begin{table}[h]
\caption{An Example of a Table}
\label{table_example}
\begin{center}
\begin{tabular}{|c||c|}
\hline
One & Two\\
\hline
Three & Four\\
\hline
\end{tabular}
\end{center}
\end{table}


  



\section{CONCLUSIONS}


\addtolength{\textheight}{-12cm}   % This command serves to balance the column lengths
                                  % on the last page of the document manually. It shortens
                                  % the textheight of the last page by a suitable amount.
                                  % This command does not take effect until the next page
                                  % so it should come on the page before the last. Make
                                  % sure that you do not shorten the textheight too much.

%%%%%%%%%%%%%%%%%%%%%%%%%%%%%%%%%%%%%%%%%%%%%%%%%%%%%%%%%%%%%%%%%%%%%%%%%%%%%%%%



%%%%%%%%%%%%%%%%%%%%%%%%%%%%%%%%%%%%%%%%%%%%%%%%%%%%%%%%%%%%%%%%%%%%%%%%%%%%%%%%



%%%%%%%%%%%%%%%%%%%%%%%%%%%%%%%%%%%%%%%%%%%%%%%%%%%%%%%%%%%%%%%%%%%%%%%%%%%%%%%%

\section*{ACKNOWLEDGMENT}




\begin{thebibliography}{99}

\bibitem{c1} G. O. Young, ÒSynthetic structure of industrial plastics (Book style with paper title and editor),Ó 	in Plastics, 2nd ed. vol. 3, J. Peters, Ed.  New York: McGraw-Hill, 1964, pp. 15Ð64.
\bibitem{c2} W.-K. Chen, Linear Networks and Systems (Book style).	Belmont, CA: Wadsworth, 1993, pp. 123Ð135.
\bibitem{c3} H. Poor, An Introduction to Signal Detection and Estimation.   New York: Springer-Verlag, 1985, ch. 4.
\bibitem{c4} B. Smith, ÒAn approach to graphs of linear forms (Unpublished work style),Ó unpublished.
\bibitem{c5} E. H. Miller, ÒA note on reflector arrays (Periodical styleÑAccepted for publication),Ó IEEE Trans. Antennas Propagat., to be publised.
\bibitem{c6} J. Wang, ÒFundamentals of erbium-doped fiber amplifiers arrays (Periodical styleÑSubmitted for publication),Ó IEEE J. Quantum Electron., submitted for publication.
\bibitem{c7} C. J. Kaufman, Rocky Mountain Research Lab., Boulder, CO, private communication, May 1995.
\bibitem{c8} Y. Yorozu, M. Hirano, K. Oka, and Y. Tagawa, ÒElectron spectroscopy studies on magneto-optical media and plastic substrate interfaces(Translation Journals style),Ó IEEE Transl. J. Magn.Jpn., vol. 2, Aug. 1987, pp. 740Ð741 [Dig. 9th Annu. Conf. Magnetics Japan, 1982, p. 301].
\bibitem{c9} M. Young, The Techincal Writers Handbook.  Mill Valley, CA: University Science, 1989.
\bibitem{c10} J. U. Duncombe, ÒInfrared navigationÑPart I: An assessment of feasibility (Periodical style),Ó IEEE Trans. Electron Devices, vol. ED-11, pp. 34Ð39, Jan. 1959.
\bibitem{c11} S. Chen, B. Mulgrew, and P. M. Grant, ÒA clustering technique for digital communications channel equalization using radial basis function networks,Ó IEEE Trans. Neural Networks, vol. 4, pp. 570Ð578, July 1993.
\bibitem{c12} R. W. Lucky, ÒAutomatic equalization for digital communication,Ó Bell Syst. Tech. J., vol. 44, no. 4, pp. 547Ð588, Apr. 1965.
\bibitem{c13} S. P. Bingulac, ÒOn the compatibility of adaptive controllers (Published Conference Proceedings style),Ó in Proc. 4th Annu. Allerton Conf. Circuits and Systems Theory, New York, 1994, pp. 8Ð16.
\bibitem{c14} G. R. Faulhaber, ÒDesign of service systems with priority reservation,Ó in Conf. Rec. 1995 IEEE Int. Conf. Communications, pp. 3Ð8.
\bibitem{c15} W. D. Doyle, ÒMagnetization reversal in films with biaxial anisotropy,Ó in 1987 Proc. INTERMAG Conf., pp. 2.2-1Ð2.2-6.
\bibitem{c16} G. W. Juette and L. E. Zeffanella, ÒRadio noise currents n short sections on bundle conductors (Presented Conference Paper style),Ó presented at the IEEE Summer power Meeting, Dallas, TX, June 22Ð27, 1990, Paper 90 SM 690-0 PWRS.
\bibitem{c17} J. G. Kreifeldt, ÒAn analysis of surface-detected EMG as an amplitude-modulated noise,Ó presented at the 1989 Int. Conf. Medicine and Biological Engineering, Chicago, IL.
\bibitem{c18} J. Williams, ÒNarrow-band analyzer (Thesis or Dissertation style),Ó Ph.D. dissertation, Dept. Elect. Eng., Harvard Univ., Cambridge, MA, 1993. 
\bibitem{c19} N. Kawasaki, ÒParametric study of thermal and chemical nonequilibrium nozzle flow,Ó M.S. thesis, Dept. Electron. Eng., Osaka Univ., Osaka, Japan, 1993.
\bibitem{c20} J. P. Wilkinson, ÒNonlinear resonant circuit devices (Patent style),Ó U.S. Patent 3 624 12, July 16, 1990. 






\end{thebibliography}




\end{document}
