\documentclass[12pt, a4paper]{article}

%% Language and font encodings
\usepackage[T2A]{fontenc}
\usepackage[utf8]{inputenc}
\usepackage[russian, english]{babel}

%% Sets page size and margins
\usepackage[a4paper,top=3cm,bottom=2cm,left=3cm,right=3cm,marginparwidth=1.75cm]{geometry}

%% Useful packages
\usepackage{amsmath}
\DeclareMathOperator*{\argmax}{argmax}
\DeclareMathOperator*{\argmin}{argmin}
\usepackage{graphicx}
\usepackage[colorinlistoftodos]{todonotes}
\usepackage[colorlinks=true, allcolors=blue]{hyperref}

\begin{document}
\selectlanguage{russian}
\title{\textbf{Определение местоположения по сигналам акселерометра}}
\author{Фатеев~Д.\,А.}
\date{}
\maketitle


\begin{abstract}
	Данная статья посвящена методам отслеживания местоположения человека по сигналам акселерометра, гироскопа, манометра. Основной задачей исследования является увеличение точности позиционирования. В качестве базовой модели был выбран PDR (pedestrian dead reckoning). Новизна исследования заключается в постановке задачи в терминах Projection to Latent Spaces.

\end{abstract}

{\bf Ключевые слова:} Pedestrian dead reckoning, (Indoor) inertial positioning, Simultaneous Localization and Mapping, PLS

\setcounter{secnumdepth}{0}
\section{Введение}



\begin{thebibliography}{4}
\end{thebibliography}

\end{document}
