\documentclass[12pt,twoside]{article}
\usepackage{jmlda}
%\NOREVIEWERNOTES
\title
%    [Образец оформления статьи для публикации] % Краткое название; не нужно, если полное название влезает в~колонтитул
    {Определение местоположения по сигналам акселерометра}
\author
%    [] % список авторов для колонтитула; не нужен, если основной список влезает в колонтитул
    {Макаров~М.\,В.} % основной список авторов, выводимый в оглавление
%    [] % список авторов, выводимый в заголовок; не нужен, если он не отличается от основного
%\thanks
%    {Работа выполнена при финансовой поддержке РФФИ, проект \No\,00-00-00000.
%   Научный руководитель:  Стрижов~В.\,В.
%   Задачу поставил:  Эксперт~И.\,О.
%    Консультант:  Консультант~И.\,О.}
%\email
%    {author@site.ru}
%\organization
%    {$^1$Организация; $^2$Организация}
\abstract
    {Мы рассматриваем задача определения местоположения человека по данным акселерометра телефона и другим вспомогательным данным. 
    Уже существуют некоторые методы решения данной задачи \cite{journals/corr/abs-1712-09004}.
    В данной статье акцент делается на использовании дополнительной информации, например сигналов гироскопа или магнетометра, 
    для повышения точности. 
%\bigskip
%\textbf{Ключевые слова}: \emph {ключевое слово, ключевое слово,
%еще ключевые слова}.
}

\begin{document}
\maketitle
%\linenumbers

\section{Введение}
В данной работе рассматривается задача определения местоположения человека по данным акселерометра его телефона. Данная задача актуальна как часть
более общей проблемы определения местоположения. Поскольку акселерометры энергоэффективны и не требуют для работы наличие внешних устройств, таких как спутник или радиоточка, точные методы решения этой задачи востребованы.
    
В силу того, что акселерометры, использующиеся в мобильных устройствах, неточны, н
наивное решение поставленной задачи путём двойного интегрирования даёт путь, значительно отклоняющийся от истинной траектории.
В \cite{journals/corr/abs-1712-09004} эта проблема решается использованием информации о том, где находится телефон во время 
перемещения человека.

Местоположение также можно определить, используя данные других датчиков, таких как магнитометр \cite{6987239} и гироскоп \cite{s18051391}.

В данной работе рассматривается метод, опирающийся как на априорные знания о положении телефона, так и на данные с магнитометра и гироскопа.
Это позволяет использовать фильтр Маджвика для предобработки сигналов. После чего для вычисления скорректированных значений вектора ускорения
используется метод PLS.

\bibliographystyle{unsrt}
\bibliography{../Project26}

\section{Постановка задачи}
Данные с датчиков представляются виде временного ряда 
$s = \{ (\mathbf{a}(t), \mathbf{w}(t), \mathbf{b}(t)) | t \in T \} \in {\mathbb{R}^{9}}^T = \mathbb{X}$, 
составленного из показаний акселерометра, гироскопа и магнитометра по 3 координатам, 
где 
$T = \{ t_1, \ldots, t_m \}$ --- множество моментов в которые проводились измерения. 
Аналогично, ряд истинных положений объекта имеет вид $y \in {\mathbb{R}^{3}}^T = \mathbb{Y}$.
Обозначим через $\mathbf{X} = \{ s_i | i \in \mathcal{I} \} $ матрицу всех рядов выборки, 
а через $\mathbf{y} = \{ y_i | i \in \mathcal{I} \}$ --- матрицу положений объекта в соответствующие моменты времени.

Предпологается, что
\begin{itemize}
    \item В момент времени $t_1$ базис системы отсчёта объекта совпадает с системой отсчёта, относительно которой происходят измерения перемещения.
    \item Погрешности измерений $\mathbf{a}_i(t), \mathbf{w}_i(t), \mathbf{b}_i(t)$ независимы и имеют нулевое матожидание.
    \item Каждый элемент выборки был получен при фиксированном расположении смартфона на теле человека --- в сумке, в руке, на теле или на ноге 
    --- соответственно классы $ P = \{0, 1, 2, 3 \} $.
\end{itemize}

Требуется найти такую модель 
$$ 
f: \mathbb{X} \rightarrow \mathbb{Y}, f(s) = \hat{y} 
$$
что среднеквадратичная функция ошибки 
$$
    S(\hat{\mathbf{y}}, \mathbf{y}) = \frac{1}{|I|} \sum\limits_{i \in \mathcal{I}} \sum\limits_{t \in T} || y_i(t) - \hat{y}_i(t) ||^2
$$ 
минимальна.

В данной работе рассматриваются модели следующего вида:
с помощью фильтра Маджвика $M$ получается ряд сглаженных значения вектора ускорения, $M(s) = \hat{a} $, после чего c помощью вспомогательной
модели $g: \mathbb{X} \rightarrow P, g(\hat{a}) = p$ определяется, к какому классу расположения относится данная траектория. После чего
используется модель $f_p(\hat{a}) = \tilde{a} $ дальнейшего уточнения вектора ускорения. Наконец, ускорения $\tilde{a}$ дважды интегрируются для 
получения оценки $\hat{y}$. Обозначим это преобразование за $\mathtt{I}$. Для моделей $f_p, g$ используется метод опорных векторов.

Таким образом, определяя $\tilde{f}(s) = I(f_{g(M(s))}(M(s)))$ формальная постановка задачи такова:
$$
    \mathbf{f}^* = \argmin_{g, f_1, \ldots, f_3 } S(\tilde{f}(\mathbf{X}), \mathbf{y})
$$

\end{document}
