\documentclass{article}
\usepackage{amsmath,amsthm,amssymb}
\usepackage{mathtext}
\usepackage[T1,T2A]{fontenc}
\usepackage[utf8]{inputenc}
\usepackage[english,bulgarian,ukranian,russian]{babel}
\usepackage[utf8]{inputenc}
\usepackage{tikz}


\title{Определение местоположения по сигналу акселерометра}


\author{\bf \em Zaynulina E. T., Kiseleva E. A., Protasov V. P., Fateev D. A.,\\
 \bf \em Bozhedomov N., Tolkanev A. A., Nochevkin V., Ryabov A. }

\date{October 2018}

\begin{document}

\maketitle

\section{Introduction}
Данная статья посвящена использованию методов машинного обучения в задаче определения
местоположения по результатам прибора, который несет движущийся
человек. Задача является актуальной и имеет множество практических применений, как,
например, автоматическое включение/выключение энергозатратных сервисов при различном
положении мобильного устройства. Поставленная задача решается по сигналам датчика телефона – акселерометра. Основной смысл работы – это способ
выбора и предобработки признаков, позволяющий уменьшить влияние шума на результат
классификации и анализировать активность в независимости от пространственной
ориентации мобильного устройства. Результаты, полученные в ходе вычислительного эксперимента,
подтверждают применимость предложенного подхода.


{\bf Новизна:} задача исследования ставится в терминах projection to the latent space (It is pretty much used that way in machine learning — you observe some data which is in the space that you can observe, and you want to map it to a latent space where similar data points are closer together.)



{\bf Ключевые слова:} обработка сигналов; сенсоры; акселерометр; анализ данных; машинное
обучение.
\section{Abstract}
Существуют много публикаций, которые посвящены задаче классификации вида физической
активности человека и идентификации по походке.

Задача определения местоположения произвольного телефона для любого пользователя
является сложной по следующим причинам: манера движения, в частности походка,
у людей сильно различается; характеристики одежды, карманов и сумок варьируются
в широких пределах, ориентация телефона в пространстве может быть произвольной.
Датчики мобильных устройств имеют значительный разброс параметров

\end{document}
