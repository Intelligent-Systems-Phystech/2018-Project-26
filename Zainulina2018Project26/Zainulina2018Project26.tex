\documentclass[12pt,twoside]{article}
\usepackage{jmlda}
%\NOREVIEWERNOTES
\title
    [Определение местоположения по сигналам акселерометра] 
    {Определение местоположения по сигналам акселерометра}
\author
    [Зайнулина~Э.\,Т.] % список авторов для колонтитула; не нужен, если основной список влезает в колонтитул
    {Зайнулина~Э.\,Т., Киселёва~Е.\,А.,Фатеев~Д.\,А.,
    Божедомов~Н., Толканев~А.\,А., Ночевкин~В.,
    Протасов~В., Рябов~А.} % основной список авторов, выводимый в оглавление
    %[Автор~И.\,О.$^1$, Соавтор~И.\,О.$^2$, Фамилия~И.\,О.$^2$] % список авторов, выводимый в заголовок; не нужен, если он не отличается от основного
\thanks
    {
   Научный руководитель:  Стрижов~В.\,В.
    Консультант:  Мотренко~А.}
%\email
%    {zaynulina.et@phystech.edu}
%\organization
 %   {$^1$Московский физико-технический институт(ГУ)}
\abstract
    { {\textbf{Аннотация:}} данная статья посвящена методам отслеживания местоположения человека по сигналам акселерометра, гироскопа, магнитометра. Основной задачей исследования является увеличение точности позиционирования в условиях, когда глобальная навигационная система не может быть использована. В качестве базовой модели был выбран PDR (pedestrian dead reckoning). Для уменьшения зашумленности данных был использован фильтр Калмана. Новизна исследования заключается в постановке задачи в терминах Projection to Latent Spaces.

\bigskip
\textbf{Ключевые слова}: \emph {Pedestrian dead reckoning, (Indoor) inertial positioning, Simultaneous Localization and Mapping, PLS}.}

\begin{document}
\maketitle
%\linenumbers

\section{Введение}
В настоящее время системы по определению местоположения человека стали неотъемлемой частью повседневной жизни. Информация о точном местоположении человека используется для обеспечения безопасности, для "мобильного здоровья", для эффективной организации рабочих процессов, для мониторинга толпы и др. Огромную роль в определении местоположения человека играет GNSS (глобальная навигационная система). Однако в помещении навигационные спутниковые сигналы не всегда доступны, из-за чего качество данных, предоставляемых GNSS, сильно уменьшается. Тем не менее большую часть времени человек проводит в помещениях, в связи с чем должны быть разработаны надежные, точные методы, позволяющие определять местоположение человека в помещении.

Современные смартфоны обладают большим числом сенсоров и высокой вычислительной способностью. Так как в настоящее время почти каждый человек им обладает, то методы определения местоположения человека с использованием смартфонов получили наибольшее внимание со стороны исследователей. Среди этих методов - методы, основанные на беспроводных сигналах (WiFi, Bluetooth, UWB) \cite{journals/puc/VeraOA11} \cite{journals/puc/KimJP13}, датчиках обзора (лазерный сканер, монокулярная и бинокулярная камера) \cite{journals/puc/BrunsB09}, инерционных датчиках (акселерометр, гироскоп, магнитометр) \cite{journals/puc/ParkSC13} \cite{journals/puc/HardeggerRT15} \cite{journals/sensors/WangLYJG18} \cite{6987239}. Многие из предложенных методов локализации человека представляют собой комбинацию выше перечисленных для увеличения точности позиционирования \cite{journals/ejasp/EvennouM06} \cite{6834746} \cite{7021969}. Методы, основанные на беспроводных сигналах и датчиках обзора, помимо наличия смартфона требуют также введения дополнительного оборудования либо наличия дополнительных знаний, например карты помещения или базы данных силы сигнала (RSSI) WiFi точки в зависимости от координаты (WiFi fingerprint). Однако не всегда возможно предоставить карту помещения, например, в силу конфиденциальности; вспомогательное оборудование, в свою очередь, требует технического обслуживания и больших затрат. Что касается WiFi позиционирования, то при наличии существующей базы данных WiFi fingerprint при некотором изменении среды, позиционирование будет неточным, поэтому база данных нуждается в постоянном обновлении \cite{journals/sensors/Torres-Sospedra17a}.

Чтобы избежать данных проблем, предлагается метод, основанный на инерционных датчиках. В качестве базового алгоритма рассматривается pedestrian dead reckoning (PDR) \cite{7743695}. По сравнению с методами, основанными на беспроводных сигналах и датчиках обзора, PDR рассчитывает относительно точное местоположение человека быстрее и потребляя меньше вычислительной мощности. Для фильтрации шума в данных используется фильтр Калмана \cite{journals/corr/abs-1712-09004}. Особенность данной работы состоит в том, чтобы восстанавливать траекторию не от точки к точке, а всю целиком. Для работы с полученным многомерным пространством предлагается использовать метод PLS \cite{10.1007/11752790_2}.

\bibliographystyle{plain}
\bibliography{Zainulina2018Project26}

\end{document}
